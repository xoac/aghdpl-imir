\chapter{Wprowadzenie}\label{cha:wprowadzenie}

\LaTeX~jest systemem składu umożliwiającym tworzenie dowolnego typu dokumentów (w~szczególności naukowych i~technicznych) o~wysokiej jakości typograficznej (\cite{Dil00}, \cite{Lam92}). Wysoka jakość składu jest niezależna od rozmiaru dokumentu --- zaczynając od krótkich listów do bardzo grubych książek. \LaTeX~automatyzuje wiele prac związanych ze składaniem dokumentów np.: referencje, cytowania, generowanie spisów (treści, rysunków, symboli itp.) itd.

\LaTeX~jest zestawem instrukcji umożliwiających autorom skład i wydruk ich prac na najwyższym poziomie typograficznym. Do formatowania dokumentu \LaTeX~stosuje \TeX{}a (wymawiamy [tech] --- greckie litery $\tau$, $\epsilon$, $\chi$). Korzystając z~systemu składu \LaTeX~mamy za zadanie przygotować jedynie tekst źródłowy, cały ciężar składania, formatowania dokumentu przejmuje na siebie system.

%---------------------------------------------------------------------------

\section{Cele pracy}\label{sec:celePracy}


Celem poniższej pracy jest zapoznanie studentów z systemem \LaTeX~w zakresie umożliwiającym im samodzielne, profesjonalne złożenie pracy dyplomowej w systemie \LaTeX.

\subsection{Jakiś tytuł}

\subsubsection{Jakiś tytuł w subsubsection}


\subsection{Jakiś tytuł 2}

%---------------------------------------------------------------------------

\section{Zawartość pracy}\label{sec:zawartoscPracy}

W rozdziale~\ref{cha:pierwszyDokument} przedstawiono podstawowe informacje dotyczące struktury dokumentów w \LaTeX{}u. Alvis~\cite{Alvis2011} jest językiem



\chapter{Test}
% BEGIN section 2
\begin{tikzpicture}[scale=1]
\tikzset{
    pics/circle vertically split/.style 2 args = {
       code = {
         \pgfmathsetmacro{\widthOne}{width("#1")+4pt}
         \pgfmathsetmacro{\widthTwo}{width("#2")+4pt}

         \node[text=green](-this_is_currcent_center){+};
         \node[xshift=-\widthOne/2] (-left) {#1};
         \node[xshift=\widthTwo/2] (-right) {#2};
         \node[fit=(-left)(-right),draw,circle,text=red](-shape) {+};
         \node(-splitline) at ($ (-left.east)!.5!(-right.west) $) {};
         \draw (-shape.north east -| -splitline.center) -- (-shape.south east -| -splitline.center);
       }
    }
}
% this doesn't center correctly but it's not part of this question
\pic[inner sep = 1pt, align=left] (A) {circle vertically split={$Aaaaaa$}{$B$}};
\pic[inner sep = 1pt, align=left, right = of A-shape] (B) {circle vertically split={A}{$Bbbbb$}};
\end{tikzpicture}
% END section 2


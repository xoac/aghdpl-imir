\documentclass[pl,12pt]{aghdpl}
% \documentclass[en,11pt]{aghdpl}  % praca w języku angielskim

% Lista wszystkich języków stanowiących języki pozycji bibliograficznych użytych w pracy.
% (Zgodnie z zasadami tworzenia bibliografii każda pozycja powinna zostać utworzona zgodnie z zasadami języka, w którym dana publikacja została napisana.)
\usepackage[english,polish]{babel}

% Użyj polskiego łamania wyrazów (zamiast domyślnego angielskiego).
\usepackage{polski}

\usepackage[utf8]{inputenc}

% Załączniki

\usepackage[toc, page]{appendix}
\renewcommand\appendixpagename{Załączniki}
\renewcommand\appendixtocname{Załączniki}

% dodatkowe pakiety

\usepackage{mathtools}
\usepackage{amsfonts}
\usepackage{amsmath}
\usepackage{amsthm}
\usepackage{float}% do umieszczenia floatów [H]
\usepackage{enumitem}
\setlist{nosep} % or \setlist{noitemsep} to leave space around whole list
\usepackage[bookmarks,hidelinks]{hyperref}

% BEGIN section 1
\usepackage{tikz}
\usetikzlibrary{calc, positioning,fit}
% END section 1

% Środowisko float do kodu źródłowego \begin{program}

\floatstyle{plaintop}
\ifcsname{chapter}\endcsname%
    \newfloat{program}{!tbh}{lop}[chapter]
\else%
    \newfloat{program}{!tbh}{lop}
\fi
\floatname{program}{Kod źr.}

% Kod poniżej powoduje, że floaty nie wylatują poza granice sekcji

\usepackage{placeins}

\ifcsname{chapter}\endcsname%
    \let\Oldchapter\chapter%
    \renewcommand{\chapter}{\FloatBarrier\Oldchapter}
\fi

\let\Oldsection\section%
\renewcommand{\section}{\FloatBarrier\Oldsection}

\let\Oldsubsection\subsection%
\renewcommand{\subsection}{\FloatBarrier\Oldsubsection}

\let\Oldsubsubsection\subsubsection%
\renewcommand{\subsubsection}{\FloatBarrier\Oldsubsubsection}

% --- < bibliografia > ---


\usepackage[
style=numeric,
sorting=none,
%
% Zastosuj styl wpisu bibliograficznego właściwy językowi publikacji.
language=autobib,
autolang=other,
% Zapisuj datę dostępu do strony WWW w formacie RRRR-MM-DD.
urldate=iso,
seconds=true,
% Nie dodawaj numerów stron, na których występuje cytowanie.
backref=false,
% Podawaj ISBN.
isbn=true,
% Nie podawaj URL-i, o ile nie jest to konieczne.
url=false,
%
% Ustawienia związane z polskimi normami dla bibliografii.
maxbibnames=3,
% Jeżeli używamy Bibera:
backend=biber
]{biblatex}

\usepackage{csquotes}
% Ponieważ `csquotes` nie posiada polskiego stylu, można skorzystać z mocno zbliżonego stylu chorwackiego.
\DeclareQuoteAlias{croatian}{polish}

\addbibresource{bibliografia.bib}

% Przecinki zamiast kropek do oddzielenia pól wpisu bibliograficznego
% i dwukropek po nazwisku autora, bez kropki na końcu
\AtBeginBibliography{
    \renewcommand\labelnamepunct{:\space}
    \renewcommand\newunitpunct{\addcomma\space}
    \renewcommand{\finentrypunct}{}

    \renewcommand{\bibopenparen}{\addcomma\addspace}
    \renewcommand{\bibcloseparen}{\addspace}
}

% Nie wyświetlaj wybranych pól.
%\AtEveryBibitem{\clearfield{note}}


% ------------------------
% --- < listingi > ---

% Użyj czcionki kroju Times.
\usepackage{newtxtext}
\usepackage{newtxmath}

\usepackage{listings}
\lstset{language=TeX}

\lstset{%
        literate={ą}{{\k{a}}}1
           {ć}{{\'c}}1
           {ę}{{\k{e}}}1
           {ó}{{\'o}}1
           {ń}{{\'n}}1
           {ł}{{\l{}}}1
           {ś}{{\'s}}1
           {ź}{{\'z}}1
           {ż}{{\.z}}1
           {Ą}{{\k{A}}}1
           {Ć}{{\'C}}1
           {Ę}{{\k{E}}}1
           {Ó}{{\'O}}1
           {Ń}{{\'N}}1
           {Ł}{{\L{}}}1
           {Ś}{{\'S}}1
           {Ź}{{\'Z}}1
           {Ż}{{\.Z}}1
}

% Ustawienia pakietu lstlisting do umieszczania kodu

\usepackage{color}

\definecolor{mygreen}{rgb}{0,0.6,0}
\definecolor{mygray}{rgb}{0.5,0.5,0.5}
\definecolor{mymauve}{rgb}{0.58,0,0.82}

\lstset{%
  backgroundcolor=\color{white},     % choose the background color
  basicstyle=\ttfamily\footnotesize, % size of fonts used for the code
  breaklines, breakatwhitespace,     % automatic line breaking only at whitespace
  commentstyle=\color{mygreen},      % comment style
  numbers=left,
  showstringspaces=false,
  numberstyle=\tiny,
  frame=l,
  escapeinside={*@}{@*},           % if you want to add LaTeX within your code
  keywordstyle=\color{blue},         % keyword style
  stringstyle=\color{mymauve}        % string literal style
}

% ------------------------

\AtBeginDocument{%
        \renewcommand{\tablename}{Tab.}
        \renewcommand{\figurename}{Rys.}
}

% ------------------------
% --- < tabele > ---

\usepackage{array}
\usepackage{tabularx}
\usepackage{multirow}
\usepackage{booktabs}
\usepackage{makecell}
\usepackage[flushleft]{threeparttable}

% defines the X column to use m (\parbox[c]) instead of p (`parbox[t]`)
\newcolumntype{C}[1]{>{\hsize=#1\hsize\centering\arraybackslash}X}


%---------------------------------------------------------------------------

\author{{[Imiona i Nazwisko]}}

\makeatletter% Poniższe makra są wyłącznie zdefiniowane w klasie aghdpl-imir
\@ifclassloaded{aghdpl}{%

  \sex{m} % Rodzaj osobowych form czasowników: m - męski, ż - żeński
  \shortauthor{{[Im.\ i nazwisko]}}
  \albumnum{{[xxxxxx]}}
  \address{{[Adres]}}

  \titlePL{{[Bardzo długi temat niezwykle dogłębnej pracy dyplomowej inżynierskiej debatującej nad ekstraordynaryjnie pasjonującym zagadnieniem]}}
  \titleEN{{[Very long title of extremely indepth engineer dimploma thesis exploring an extraordinarily interesing subject]}}

  \shorttitlePL{{[Skrócony temat pracy]}} % skrócona wersja tytułu
  \shorttitleEN{{[Short thesis subject]}}

  % rodzaj pracy bez końcówki fleksyjnej np. inżyniersk, magistersk
  \thesistypePL{magistersk}
  \thesistypeEN{engineer}

  \supervisor{{[Tytuł, imię i nazwisko promotora]}}

  \reviewer{{[Tytuł, imię i nazwisko recenzenta]}}

  \degreeprogrammePL{{[Nazwa kierunku studiów]}}
  \degreeprogrammeEN{{[Field of Study]}}

  \specialisationPL{{[Nazwa specjalności]}}
  \specialisationEN{{[Specialisation]}}

  \graduationyear{{[Rok ukończenia]}}
  \years{2017/2018}
  \yearofstudy{IV}
  \formPL{stacjonarne}
  \formEN{full-time}

  % zgoda na publikację pracy w internecie: t-zgoda, cokolwiek
  % innego-brak zgody
  \agree{t}

  % praktyka (dyplomowa)
  \apprenticeship{{[Praktyka dyplomowa]}}

  \department{Katedra Transportu Linowego}

  \facultyPL{Wydział Inżynierii Mechanicznej i Robotyki}
  \facultyEN{Faculty of Mechanical Engineering and Robotics}

  \thesisplan{% Przykładowy plan pracy, należy omówić z promotorem
    \begin{enumerate}
    \item Omówienie tematu pracy i sposobu realizacji z promotorem.
    \item Zebranie i opracowanie literatury dotyczącej tematu pracy.
    \item Zebranie i opracowanie wyników badań.
    \item Analiza wyników badań, ich omówienie i zatwierdzenie przez promotora.
    \item Opracowanie redakcyjne.
    \end{enumerate}
  }

  \summaryPL{\indent\indent%
	  {[Treść streszczenia]}
  }
  \summaryEN{\indent\indent%
	  {[Summary text]}
  }

  \acknowledgements{%
    Serdecznie dziękuję \dots tu ciąg dalszych podziękowań np.\ dla promotora,
    żony, sąsiada itp.
  }

  \setlength{\cftsecnumwidth}{10mm}
}{}%
\makeatother%

\date{\today}

%---------------------------------------------------------------------------
\setcounter{secnumdepth}{4}
\brokenpenalty=10000\relax

\begin{document}

\titlepages{}

% Ponowne zdefiniowanie stylu `plain`, aby usunąć numer strony z pierwszej strony spisu treści i poszczególnych rozdziałów.
\fancypagestyle{plain}
{%
        % Usuń nagłówek i stopkę
        \fancyhf{}
        % Usuń linie.
        \renewcommand{\headrulewidth}{0pt}
        \renewcommand{\footrulewidth}{0pt}
}

\setcounter{tocdepth}{2}
{\singlespacing\tableofcontents}
\clearpage

\include{rozdzial3}
\chapter{Wprowadzenie}\label{cha:wprowadzenie}

\LaTeX~jest systemem składu umożliwiającym tworzenie dowolnego typu dokumentów (w~szczególności naukowych i~technicznych) o~wysokiej jakości typograficznej (\cite{Dil00}, \cite{Lam92}). Wysoka jakość składu jest niezależna od rozmiaru dokumentu --- zaczynając od krótkich listów do bardzo grubych książek. \LaTeX~automatyzuje wiele prac związanych ze składaniem dokumentów np.: referencje, cytowania, generowanie spisów (treści, rysunków, symboli itp.) itd.

\LaTeX~jest zestawem instrukcji umożliwiających autorom skład i wydruk ich prac na najwyższym poziomie typograficznym. Do formatowania dokumentu \LaTeX~stosuje \TeX{}a (wymawiamy [tech] --- greckie litery $\tau$, $\epsilon$, $\chi$). Korzystając z~systemu składu \LaTeX~mamy za zadanie przygotować jedynie tekst źródłowy, cały ciężar składania, formatowania dokumentu przejmuje na siebie system.

%---------------------------------------------------------------------------

\section{Cele pracy}\label{sec:celePracy}


Celem poniższej pracy jest zapoznanie studentów z systemem \LaTeX~w zakresie umożliwiającym im samodzielne, profesjonalne złożenie pracy dyplomowej w systemie \LaTeX.

\subsection{Jakiś tytuł}

\subsubsection{Jakiś tytuł w subsubsection}


\subsection{Jakiś tytuł 2}

%---------------------------------------------------------------------------

\section{Zawartość pracy}\label{sec:zawartoscPracy}

W rozdziale~\ref{cha:pierwszyDokument} przedstawiono podstawowe informacje dotyczące struktury dokumentów w \LaTeX{}u. Alvis~\cite{Alvis2011} jest językiem



\chapter{Test}
% BEGIN section 2
\begin{tikzpicture}[scale=1]
\tikzset{
    pics/circle vertically split/.style 2 args = {
       code = {
         \pgfmathsetmacro{\widthOne}{width("#1")+4pt}
         \pgfmathsetmacro{\widthTwo}{width("#2")+4pt}

         \node[text=green](-this_is_currcent_center){+};
         \node[xshift=-\widthOne/2] (-left) {#1};
         \node[xshift=\widthTwo/2] (-right) {#2};
         \node[fit=(-left)(-right),draw,circle,text=red](-shape) {+};
         \node(-splitline) at ($ (-left.east)!.5!(-right.west) $) {};
         \draw (-shape.north east -| -splitline.center) -- (-shape.south east -| -splitline.center);
       }
    }
}
% this doesn't center correctly but it's not part of this question
\pic[inner sep = 1pt, align=left] (A) {circle vertically split={$Aaaaaa$}{$B$}};
\pic[inner sep = 1pt, align=left, right = of A-shape] (B) {circle vertically split={A}{$Bbbbb$}};
\end{tikzpicture}
% END section 2


\include{rozdzial2}
\include{tests}

%Kod poniżej dodaje Bibliografię do spisu treści
\cleardoublepage{}
\phantomsection{}
\addcontentsline{toc}{chapter}{Bibliografia}
\printbibliography{}

% Załączniki
\begin{appendices}
  \makeatletter% Kod poniżej powoduje ustęp w spisie treści
  \addtocontents{toc}{\let\protect\l@chapter\protect\l@section}
  \input{zalocznikA}
  % \input{zalocznikB}
  % itd.
  \makeatother%
\end{appendices}

\end{document}
